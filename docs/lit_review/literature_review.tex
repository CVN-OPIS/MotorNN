\documentclass[conference]{IEEEtran}
\IEEEoverridecommandlockouts
% The preceding line is only needed to identify funding in the first footnote. If that is unneeded, please comment it out.
\usepackage{cite}
\usepackage{amsmath,amssymb,amsfonts}
\usepackage{algorithmic}
\usepackage{graphicx}
\usepackage{textcomp}
\usepackage{booktabs, times, epsfig, graphicx, amsmath, amssymb,url, multirow,comment,commath}


\begin{document}

\title{Literature Review: Machine Learning Applied to Dynamic Physial System}
\date{August 2018}

\maketitle

\section{Abstract}
This survery is on recent advancements in the intersection of physical modelling and machine learning.

\section{Introduction}
Understanding of physical process from data when there is no first prinicple solution is very hard. The abundance of data in both natural and physical sciences has enabled the use of machine learning models to understand governing dynamics of many complex processes. There are different ways in which physical modeling and machine learning methods have been used together. There are works in the problem of understanding pysical process from data, classifying or predicting complex physical process, using physics to generate simulation data, using machine learning to control non-linear dynamical systems, using machine learning to do fault detection in dynamical systems, etc.

This survey is on all the different areas where there has been an amalgation of machine learning and physics. The application on which we focus is on time series modeling, non-linear control, motor control, and fault detection.

\section{Background}
The background section is divided in to the following subsections; first is modeling of physical systems, second is on non-linear control, third is on motor control, and fourth is on fault detection.

\subsection{Modeling of physical systems}

This section is on recent works where modeling of dynamical systems using machine learning has been done. Modeling of complex systems has been presented in \cite{langford2009nonlinear, rudy2017datadriven, levin1991nips, doya1996nips, hermans2014automated, john2009nonlinear, timothy1994nips, ardizzone2018analyzing, raissi2018deep, lusch2017deep, karpatne2017nips, d'agnolo2018learning, ogunmolu2016nonlinear, karpatne2017physics-guided, karpatne2017theory-guided, oliveria2017nips, benzenac2017dlphy}

HMM or Kalman filter are able to learn linear dynamic models, for non-linear dynamics accomodating nonlinearity into HMM is very hard. In \cite{langford2009nonlinear} a new method called sufficient posterior representation is presented which can be used to model nonlinear dynamic behaviors using many nonlinear supervised learning algorithms such as neural networks, boosting and SVM in a simple and unified fashion.

PDEs can describe complex phoenomena. We don't always have PDEs for a given problem, but we may have a large amount of data available. In \cite{rudy2017datadriven} a data driven method is proposed to learn governing PDEs of a given system from time series data. Sparse regression is used to learn the coefficients and an iterative method is used to get most suitable coefficients. Experiments on Navier-Stokes equation is shown.

In \cite{levin1991nips} modeling of time invariant nonlinear systems is addressed. A multi-layered network architecture with a control input signal called Hidden Control Neural Network (HCNN) is presented which can model signals generated by nonlinear dynamical systems with restricted time variability.

Reinforcement learning has also been used for physial modelling. In \cite{doya1996nips} a Temporal Difference learning algorithm for continuous-time, continuous-state, nonlinear control problems is presented. Kernel regression method is used to learn a nonlinear auto-regressive model.

\cite{hermans2014automated} uses machine learning to optimize physical dynamic systems.

Most of the methods of data driven learning of dynamic systems deal with sequential data. In \cite{john2009nonlinear} method is presented to learn dynamics from non-sequential data.

\subsection{Nonlinear Control}

\cite{plett2003nn, kim2005nips, aboueldahab2011identification, levin1991nips, milito1991nips, lippmann1991nips, scott1992nips, HBZnips96, takashi2005nonlinear, sabino1999chaos, sergey2011nips, schnider1997nips, doya1997nips, rawlik2010nips, emanuel2009nips, li2015dynamic, watter2015nips, jordan1990nips, rivals2000nn, brown1998ICDC, susemihl2014nips, Mozer1996TheNP, deng2008feedback, chen2002ICDC, timothy1994nips, yu1996nips}

\subsection{Motor Control}

\cite{yao2010adaline, nouri2008adaptive, aamir13pid, kumarawadu2010discrete-time, brdys1999dynamic, kim1991ICIECI, meng2016safety, Shao2017, alicmotor, zhang2017fault, silva2013fault, murphey2006fault, murphey2006model}

\subsection{Fault Detection}

\cite{meng2016safety, Shao2017, alicmotor, zhang2017fault, silva2013fault, murphey2006fault, murphey2006model, George2017qtr, marko1991nips, kim2005nips}

\bibliographystyle{IEEEtran}
\bibliography{literature_review}

\end{document}
