\documentclass[conference]{IEEEtran}
\IEEEoverridecommandlockouts
% The preceding line is only needed to identify funding in the first footnote. If that is unneeded, please comment it out.
\usepackage{cite}
\usepackage{amsmath,amssymb,amsfonts}
\usepackage{algorithmic}
\usepackage{graphicx}
\usepackage{textcomp}
\usepackage[LGR,T1]{fontenc}
\usepackage[utf8]{inputenc}
\usepackage[inline]{enumitem}
\usepackage{booktabs, times, epsfig, graphicx, amsmath, amssymb,url, multirow,comment,commath}


\begin{document}

\title{A Survey on Machine Learning Applied to Dynamic Physical Systems}
\author{Sagar Verma \\
        \textit{Centre de Vision Num\'erique,
        Centralesup\'elec and INRIA Galen Post}
        }
\date{August 2018}

\maketitle

\section{Abstract}
This survey is on recent advancements in the intersection of physical modeling and machine learning. We focus on the modeling of nonlinear systems which are closer to electric motors. Survey on motor control and fault detection in operation of electric motors has been done. \\

\section{Introduction}
Understanding of physical process from data when there is no first principle solution is very hard. The abundance of data in both natural and physical sciences has enabled the use of machine learning models to understand the governing dynamics of many complex processes. There are different ways in which physical modeling and machine learning methods have been used together. Research has been done in the following areas; \begin{enumerate*} [label=\itshape\alph*\upshape)] \item the problem of understanding the physical process from data, \item classifying or predicting complex physical process, \item using physics to generate simulation data for machine learning models, \item using machine learning to control non-linear dynamical systems, and \item using machine learning to do fault detection in dynamical systems. \end{enumerate*}

This survey is on all the different areas where there has been an amalgamation of machine learning and physics. The applications on which we focus on are time series modeling, non-linear control, motor control, and fault detection. The survey is divided into the following sections; \begin{enumerate*}[label=\itshape\roman*\upshape)]  \setcounter{enumi}{2} \item first is modeling of physical systems, \item second is on non-linear control, \item third is on motor control, and \item the fourth is on fault detection. \end{enumerate*} \\

\section{Modeling of physical systems}

\subsection{\textbf{Data-driven modeling}}

Data-driven approaches for modeling physical systems are proposed in \cite{langford2009nonlinear, levin1991nips, john2009nonlinear, ogunmolu2016nonlinear}.  Hidden Markov Model (HMM), or Kalman filter can learn linear dynamic models. For non-linear dynamics accommodating nonlinearity into HMM is very hard. In \cite{langford2009nonlinear} a new method called sufficient posterior representation is presented which can be used to model nonlinear dynamic behaviors using many nonlinear supervised learning algorithms such as neural networks, boosting and SVM in a simple and unified fashion. In \cite{levin1991nips} modeling of time-invariant nonlinear systems is addressed. A multi-layered network architecture with a control input signal called Hidden Control Neural Network (HCNN) is presented which can model signals generated by nonlinear dynamical systems with restricted time variability. Most of the methods of data-driven learning of dynamic systems deal with sequential data. In \cite{john2009nonlinear} method is presented to learn dynamics from non-sequential data. In \cite{ogunmolu2016nonlinear} three deep neural network structures are trained on sequential data to learn complex behavior.

\subsection{\textbf{Modeling using partial differential equations}}

Another approach of modeling physical system is to represent it in the form of Partial Differential Equations (PDEs). PDEs can describe complex phenomena. We don't always have PDEs for a given problem, but we may have a large amount of data available. In \cite{rudy2017datadriven} a data-driven method is proposed to learn the governing PDEs of a given system from time series data. Sparse regression is used to learn the coefficients and an iterative method is used to get the most suitable coefficients. Experiments on the Navier-Stokes equation is shown. In \cite{raissi2018deep} a deep learning approach for discovering nonlinear partial differential equations from scattered and potentially noisy observations is presented. Two deep neural networks are used to approximate solution and nonlinear dynamics.

\subsection{\textbf{Representation learning of dynamical systems}}

For some physical systems, we want a weak governing dynamics in form of equations. A neural network trained on physical system data does not provide a good representation in form of equations. \cite{ardizzone2018analyzing, lusch2017deep} have shown methods which can be used to find weak governing dynamics in form of equations or sparse matrices. Computing hidden system parameters from measurable quantities of complex physical systems using an Invertible Neural Network (INN) is presented in \cite{ardizzone2018analyzing}. In \cite{lusch2017deep} a data-driven approach of approximating nonlinear dynamics to a linear one using deep neural networks has been present. Koopman operators \cite{Koopman315} are learned from data for coordinate transformation of a nonlinear system to a linear one. Koopman operator \cite{Koopman315} is a linear operator $C_{\phi}$ defined by the rule $C_{\phi}(f) = f \circ \phi$, where $f \circ \phi$ denotes function composition. Other methods for nonlinear to a linear transformation are presented in \cite{d'agnolo2018learning, levin1991nips}. There are systems where dynamics change with time and some dynamics may not have been seen before. Identifying new dynamics will be useful. \cite{d'agnolo2018learning} uses neural networks to identify new physics. In \cite{levin1991nips} a multi-layered neural network called Hidden Control Neural Network (HCNN) is presented to model nonlinear dynamical systems with restricted time variability. The mapping of the neural network changes with time as a function of an additional control input signal.

\subsection{\textbf{Theory-guided learning of dynamical systems}}

It is crucial to have a machine learning model which is consistent with the physics of the dynamical system. \cite{karpatne2017theory-guided} has shown how physics can be used to do better data-driven discoveries. Theory-guided design, learning, refinement of the machine learning model has been presented. In \cite{karpatne2017nips, karpatne2017physics-guided} a physics-guided neural network (PGNN) is presented which leverages the output of physics-based model simulations along with observational features to generate predictions using a neural network. The model predictions not only show lower errors on the training data but are also consistent with the system dynamics. \cite{hermans2014automated} uses machine learning to optimize physical dynamic systems.

\subsection{\textbf{Modeling using reinforcement learning}}

Reinforcement learning (RL) has also been used for physical modeling. In \cite{doya1996nips} a Temporal Difference learning algorithm for continuous-time, continuous-state, nonlinear control problems is presented. Kernel regression method is used to learn a nonlinear auto-regressive model.  \\

\section{Nonlinear Control}

\subsection{\textbf{Feedforward neural network based controllers}}

Feedforward neural networks have been used for nonlinear control in \cite{milito1991nips, scott1992nips, rawlik2010nips, Mozer1996TheNP, chen2002ICDC, yu1996nips}. In \cite{milito1991nips} a feedforward neural network is used to control an unknown stochastic nonlinear dynamical system. In \cite{scott1992nips} the governing equations of a Proportional–Integral–Derivative (PID) controller is used to train a neural network to control a nonlinear system. In \cite{rawlik2010nips} a method for jointly optimizing the temporal parameters along with the control command profiles is presented. In \cite{Mozer1996TheNP} an adaptive controller for thermostat is presented. The consequences of control decisions are delayed in time, the controller anticipates heating demands with predictive models of occupancy patterns and the thermal response of the house and furnace. Occupancy pattern prediction is achieved by a hybrid neural net/look-up table. The controller searches, at each discrete time step, for a decision sequence that minimizes the expected cost over a fixed planning horizon. The first decision in this sequence is taken, and this process repeats.

\subsection{\textbf{Stablility in neural network based controllers}}

In \cite{chen2002ICDC}  a new framework for intelligent control is presented which adaptively controls a class of nonlinear discrete-time dynamical systems while assuring boundedness of all signals. A linear robust adaptive controller and multiple nonlinear neural network based adaptive controllers are used, and a switching law is suitably defined to switch between them, based upon their performance in predicting the plant output. Boundedness of all the signals is established regardless of the parameter adjustment mechanism of the neural network controllers, and thus neural network models can be used in novel ways to better detect changes in the system and provide starting points for adaptation. In \cite{yu1996nips} a neural network based approach is presented for controlling two distinct types of nonlinear systems. The first corresponds to nonlinear systems with parametric uncertainties where the parameters occur nonlinearly. The second corresponds to systems for which stabilizing control structures cannot be determined. The proposed neural controllers are shown to result in closed-loop system stability under certain conditions.

\subsection{\textbf{Sequence-to-sequence networks based controllers}}

Most dynamical systems have time as one dimension and having a controller that can take sequence into account is very useful. \cite{timothy1994nips, plett2003nn, aboueldahab2011identification, lippmann1991nips} use Recurrent Neural Network (RNN) for nonlinear control. Using a recurrent network to create a mixture of experts for modeling and controlling dynamical systems is presented in \cite{timothy1994nips}. In \cite{plett2003nn} a dynamical system is first modeled using a recurrent neural network (RNN). Then the dynamic response of the system is controlled using another RNN. Disturbance canceling is performed using an additional RNN. In \cite{aboueldahab2011identification} a recurrent neural network architecture called Sigmoid Diagonal Recurrent Neural Network (SDRNN) is used for adaptive control of nonlinear dynamical systems. In \cite{lippmann1991nips} recurrent neural network is used to control nonlinear plants. The proposed method is used in controlling landing of a commercial aircraft in severe wind conditions.

\subsection{\textbf{Reinforcement learning based controllers}}

Reinforcement learning (RL) methods for optimal control of nonlinear systems are presented in \cite{HBZnips96, takashi2005nonlinear, sabino1999chaos, schnider1997nips, doya1996nips}. RL algorithm that learns to combine open-loop and closed-loop control is presented in \cite{HBZnips96}. Kuremoto et. al. \cite{takashi2005nonlinear} uses RL to predict nonlinear time series. In \cite{sabino1999chaos} a general purpose chaos control algorithm based on RL is introduced and applied to the stabilization of unstable periodic orbits in various chaotic systems and the targeting problem. The algorithm does not require any information about the dynamical system nor about the location of periodic orbits. Model-based RL combined with dynamic programming has been shown to be useful for learning control of continuous state dynamic systems. The simplest method assumes the learned model is correct and applies dynamic programming to it, but many approximators provide uncertainty estimates on the fit. \cite{schnider1997nips} have presented the case where the system must be prevented from having catastrophic failures during learning. In \cite{doya1997nips} a new RL architecture for nonlinear control is presented. A direct feedback controller, or the actor, is trained by a value-gradient based controller, or the tutor. This architecture enables both efficient use of the value function and simple computation for real-time implementation. Good performance was verified in multi-dimensional nonlinear control tasks using Gaussian softmax networks.

\subsection{\textbf{Convolutional neural network based controller}}

In \cite{watter2015nips} a method for model learning and control of non-linear dynamical systems from raw pixel images is presented. Embedd-to-control (E2C) consists of a deep generative model that learns to generate image trajectories from a latent space in which the dynamics are constrained to be locally linear. \\

\section{Motor Control}

\subsection{\textbf{Neural Network controller}}

In \cite{yao2010adaline} a neural network adaptive inverse controller is presented. A neural network constructs the dynamical system inverse model identifier. The task is accomplished by generating a tracking error between the input command signal and the system response. The error signal updates the weights of the neural network in such a way that the error is minimized and the neural network is close to the system inverse model. The above steps make the gain of the serial connection system close to unity, realizing waveform replication function in real-time. To enhance its convergence and robustness, normalized least mean square algorithm is applied.

\subsection{\textbf{Recurrent Neural Network controller}}

In \cite{nouri2008adaptive} a model-following adaptive control structure is proposed for the speed control of a nonlinear motor drive system and the compensation of the nonlinearities. A recurrent neural network is used for the online modeling and control of the nonlinear motor drive system with high static and Coulomb friction. The neural network is first trained off-line to learn the inverse dynamics of the motor drive system using a modified form of the decoupled extended Kalman filter algorithm. It is shown that the recurrent neural network structure combined with the inverse model control approach allows an effective direct adaptive control of the motor drive system. The performance of this method is validated experimentally on a dc motor drive system using a standard personal computer.

\subsection{\textbf{Neural Network for multi-input-multi-output control}}

In \cite{brdys1999dynamic} application of recently developed adaptive control techniques based on neural networks to the induction motor control is presented. The case study represents one of the more difficult control problems due to the complex, nonlinear, and time-varying dynamics of the motor and unavailability of full-state measurements. A partial solution is first presented based on a single-input-single-output (SISO) algorithm employing static multilayer perceptron (MLP) networks. A novel technique is subsequently described which is based on a recurrent neural network employed as a dynamical model of the plant. Recent stability results for this algorithm are reported. The technique is applied to multi-input-multi-output (MIMO) control of the motor.

\subsection{\textbf{Other Deep Neural Network based controllers}}

In \cite{aamir13pid} a deep learning controller is designed by learning a PID controller. The input/output of the PID controller is used as the learning data set for the deep learning network. Deep Belief Network algorithm is used to design the deep learning controller. In \cite{kumarawadu2010discrete-time} a discrete time neuro-compensated dynamic state feedback control system for lateral and longitudinal control of intelligent vehicle highway systems (IVHS) is presented.




\section{Fault Detection}

\subsection{\textbf{Stability guarantees}}

Learning algorithms have enjoyed numerous successes in robotic control tasks. In problems with time-varying dynamics, online learning methods have also proved to be a powerful tool for automatically tracking and/or adapting to the changing circumstances. However, for safety-critical applications such as airplane flight, the adoption of these algorithms has been significantly hampered by their lack of safety, such as “stability,” guarantees. In \cite{kim2005nips} authors have presented a method for “monitoring” the controllers suggested by the learning algorithm online, and rejecting controllers leading to instability.

\subsection{\textbf{Fault detection in electric motors}}

In \cite{silva2013fault} three different machine learning methods are presented for fault detection in electric motors. First one is feature extraction using Principal Component Analysis (PCA), the second one is classification using the k-Nearest Neighbor (k-NN) or Probabilistic Neural Network (PNN) methods and the third one is classifier performance evaluation using the Cross-Validation (CV) method. While electric machine, inverter, and sensor faults are introduced, the supervised learning algorithms are applied to four case studies where two fault modes occur in a current sensor, and two occur in the speed encoder.

\subsection{\textbf{Operating conditions as priors in fault detection}}

\cite{alicmotor} deals with the application of speed variable pumps in industrial hydraulic systems. The benefit of the natural feedback of the load torque is investigated for the issue of condition monitoring as the development of losses can be taken as evidence of faults. A neural network is used for adaptive modeling of the torque balance over a range of steady operation in fault-free behavior. The goal is to keep a numeric reference with an acceptable accuracy of the unit used in particular, taking into consideration the manufacturing tolerances and other operation conditions differences. The learned model gives a baseline for identification of significant possible abnormalities and offers a fundament for fault isolation by continuously estimating and analyzing the deviations.

\subsection{\textbf{Other methods for fault detection}}

In \cite{murphey2006fault, murphey2006model} a machine learning algorithm has been presented which automatically selects a set of representative operating points in the (torque, speed) domain, which is sent to the simulated electric drive model to generate signals for the training of a diagnostic neural network called Fault Diagnostic Neural Network (FDNN). In \cite{meng2016safety} the fault degree and health degree of the system are put forward based on the analysis of electric motor drive system’s control principle. With the self-organizing neural network’s advantage of competitive learning and unsupervised clustering, the system’s health clustering and safety identification are worked out. In \cite{Shao2017} a Deep Belief Network (DBN) is used to learn features from the frequency distribution of vibration signals with the purpose of characterizing the working status of induction motors. In \cite{zhang2017fault} DNN on raw time series data is applied to do fault detection. In \cite{marko1991nips} fault detection in real-time systems using ANN and adaptive control of an active suspension system is presented.


\bibliographystyle{IEEEtran}
\bibliography{literature_review}

\end{document}
